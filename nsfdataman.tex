\setcounter{page}{1}
\renewcommand{\thepage}{Data Management Plan - Page \arabic{page} of 1}
%\required{Supplementary Documentation}
%\required{Data Management Plan}
%\begin{center}
%\emph{Maximum of 2 pages}
%\end{center}•
% Maximum of 2 pages
%------------------------------

% This supplement should describe how the proposal will conform to NSF policy on
% the dissemination and sharing of research results and may include:
% 
% 1. The types of data, samples, physical collections, software, curriculum
%    materials, and other materials to be produced in the course of the project;
% 
% 2. The standards to be used for data and metadata format and content (where
%    existing standards are absent or deemed inadequate, this should be documented
%    along with any proposed solutions or remedies);
% 
% 3. Policies for access and sharing including provisions for appropriate
% protection   of privacy, confidentiality, security, intellectual property, or
% other rights or   requirements;
% 
% 4. Policies and provisions for re-use, re-distribution, and the production of
%    derivatives; and
% 
% 5. Plans for archiving data, samples, and other research products, and for
%    preservation of access to them.
% 
% A valid Data Management Plan may include only the statement that no detailed
% plan is needed, as long as the statement is accompanied by a clear
% justification. Proposers who feel that the plan cannot fit within the supplement
% limit of two pages may use part of the 15-page Project Description for
% additional data management information. Proposers are advised that the Data
% Management Plan may not be used to circumvent the 15-page Project Description
% limitation.

%(A-1) Sharing of Results and Management of Intellectual Property (maximum 3 pages): Describe the management of intellectual property rights related to the proposed project, including plans for sharing data, information, and materials resulting from the award. This plan must be specific about the nature of the results to be shared, the timing and means of release, and any constraints on release. The proposed plan must take into consideration the following conditions where applicable:
%
%Sequences resulting from high-throughput large-scale sequencing projects (low pass whole genome sequencing, BAC end sequencing, ESTs, full-length cDNA sequencing, etc.) must be released according to the currently accepted community standard (e.g. Bermuda/Ft. Lauderdale agreement) to public databases (GenBank if applicable), as soon as they are assembled and quality checked against a stated, pre-determined quality standard. "At publication" is not acceptable.
%Proposals that would develop genome-scale expression data through approaches such as microarrays or next-generation sequencing should meet community standards for these data (for example, Minimum Information About a Microarray Experiment or MIAME standards). The community databases (e.g. Gene Expression Omnibus) into which the data would be deposited, in addition to any project database(s) should be indicated.
%If the proposed project would produce genome-scale data sets generated using proteomics and/or metabolomics approaches, NSF encourages that they be made available as soon as their quality is checked to satisfy the specifications approved prior to funding. The timing of release should be stated clearly in the proposal. The community databases into which the data would be deposited, in addition to any project database(s) should be indicated.
%If the proposed project would produce community resources (biological materials, software, etc.), NSF encourages that they be made available as soon as their quality is checked to satisfy the specifications approved prior to funding. The timing of release should be stated clearly in the proposal. The resources produced must be available to all segments of the scientific community, including industry. A reasonable charge is permissible, but the fee structure must be outlined clearly in the proposal. If accessibility differs between industry and the academic community, the differences must be clearly spelled out. If a Material Transfer Agreement is required for release of project outcomes, the terms must be described in detail.
%When the project involves the use of proprietary data or materials from other sources, the data or materials resulting from NSF funded research must be readily available without any restrictions to the users of such data or materials (no reach-through rights). The terms of any usage agreements should be stated clearly in the proposal.
%Budgeting and planning for short-term and long-term distribution of the project outcomes must be described in the proposal. If a fee is to be charged for distribution of project outcomes, the details should be described clearly in the proposal. Letters of commitment should be provided from databases or stock centers that would distribute project outcomes, including an indication of what activities would be undertaken and funds needed for these activities (if any).
%In case of a multi-institutional proposal, the lead institution is responsible for coordinating and managing the intellectual property resulting from the PGRP award. Institutions participating in multi-institutional projects should formulate a coherent plan for the project prior to submission of the proposal.
%

SEE APPENDIX A-1 UPLOADED AS A SUPPLEMENTARY DOCUMENT