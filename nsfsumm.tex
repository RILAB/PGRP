
%%%%%%%%% PROJECT SUMMARY -- 1 page, third person
% e.g:  "The PI will prove" not "I will prove"

%Below are the pagination, font size, spacing and margin
%instructions for NSF proposals: \\
%
%FastLane does not automatically paginate a proposal.
%Each section of the proposal must be individually
%paginated prior to upload to the system. \\
%
%Use Computer Modern family of fonts at a font size of 11 points or
%larger. A font size of less than 10 points may be used for mathematical
%formulas or equations, figure, table or diagram captions and when
%using a Symbol font to insert Greek letters or special characters.
%The text must still be readable. The use of small type not in compliance with the NSF guidelines
%may be grounds for NSF to return the proposal without review. \\
%
%No more than 6 lines of text within a vertical space of 1 inch. \\
%
%Margins, in all directions, must be at least an inch. \\
%
%

\required{Project Summary}
\begin{center}
\emph{Maximum of 1 page}
\end{center}•
% This should be a brief statement of the problem you plan to address.
% It should look something like an abstract. 

%The project summary should be a description of the proposed activity suitable
%for publication, no more than one page in length. It should not be
%an abstract of the proposal, but rather a self-contained description of
%the activity that would result if the proposal were funded. The summary
%should be written in the third person and include a statement of objectives
%and methods to be employed. It should be informative to other persons
%working in the same or related fields and understandable to a scientifically
%or technically literate lay reader. \\
%
%The summary must clearly address in separate statements (within the one-page summary):
%the intellectual merit of the proposed activity; and the broader impacts
%resulting from the proposed activity. Proposals that do not separately
%address both criteria within the one-page Project Summary will be returned without
%review. \\

\required{Intellectual Merit}
% This is why your project is interesting and will help further
% knowledge in the field of mathematics. 

%How important is the proposed activity to advancing
%knowledge and understanding within its own field or across different fields?
%How well qualified is the proposer (individual or team) to conduct the project?
%(If appropriate, the reviewer will comment on the quality of prior work.)
%To what extent does the proposed activity suggest and explore creative, original,
%or potentially transformative concepts? How well conceived and organized is the
%proposed activity? Is there sufficient access to resources?  \\

\required{Broader Impacts}
% There are 4 kinds of broader impacts.
% 1. advance discovery and understanding while promoting teaching,
% training and learning
% 2. broaden the participation of underrepresented groups
% 3. disseminated broadly to enhance scientific and technological
% understanding
% 4. benefits of the proposed activity to society

