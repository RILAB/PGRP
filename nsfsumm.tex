
%%%%%%%%% PROJECT SUMMARY -- 1 page, third person
% e.g:  "The PI will prove" not "I will prove"

%Below are the pagination, font size, spacing and margin
%instructions for NSF proposals: \\
%
%FastLane does not automatically paginate a proposal.
%Each section of the proposal must be individually
%paginated prior to upload to the system. \\
%
%Use Computer Modern family of fonts at a font size of 11 points or
%larger. A font size of less than 10 points may be used for mathematical
%formulas or equations, figure, table or diagram captions and when
%using a Symbol font to insert Greek letters or special characters.
%The text must still be readable. The use of small type not in compliance with the NSF guidelines
%may be grounds for NSF to return the proposal without review. \\
%
%No more than 6 lines of text within a vertical space of 1 inch. \\
%
%Margins, in all directions, must be at least an inch. \\
%
%
\required{Project Summary}

\required{Overview}

Despite its relevance to climate change and crop improvement, the genomic architecture of local adaptation in plant populations remains poorly characterized.  In this project, the Co-PIs will investigate the genome-wide underpinnings of local adaptation in wild and domesticated populations of maize (\emph{Zea mays} ssp.) in high elevation environments.  Project collaborators will first identify quantitative trait loci for highland adaptation traits using mapping populations developed from Mexican and South American maize as well as a naturally admixed population of highland and lowland teosinte (\emph{i.e.}, wild maize).  These populations will allow for comparison of the architecture and effect sizes of highland traits in two distinct geographical regions and across subspecies.  Second, the population genetic signatures of selection on highland adaptation alleles will be assessed in three panels: a worldwide collection of highland maize, maize from Mexico that has adaptive to the highlands via introgression from teosinte, and admixed populations of  teosinte.  Finally, functional characterization will be conducted through dissection of an inversion polymorphism on chromosome four thought to play an important role in highland adaptation in Mexico and assessment of transcriptomic variation in highland- and lowland-adapted material across environments.

% This should be a brief statement of the problem you plan to address.
% It should look something like an abstract. 

%The project summary should be a description of the proposed activity suitable
%for publication, no more than one page in length. It should not be
%an abstract of the proposal, but rather a self-contained description of
%the activity that would result if the proposal were funded. The summary
%should be written in the third person and include a statement of objectives
%and methods to be employed. It should be informative to other persons
%working in the same or related fields and understandable to a scientifically
%or technically literate lay reader. \\
%
%The summary must clearly address in separate statements (within the one-page summary):
%the intellectual merit of the proposed activity; and the broader impacts
%resulting from the proposed activity. Proposals that do not separately
%address both criteria within the one-page Project Summary will be returned without
%review. \\

\required{Intellectual Merit}

Selection shapes their genomes of plants by fine-tuning them to their local biotic and abiotic conditions.  Surprisingly little is known about the consistency of this adaptive process across similar environments and the extent to which genomes are altered. Only a handful of investigations characterizing the genetic architecture and effect sizes of locally adaptive loci have been published to date and no such studies have been conducted in an economically important plant. Given the repercussions of local adaptation for conservation and agriculture in the face of climate change and human population pressure, the activities proposed here are both important and potentially transformative.  Basic evolutionary insight regarding local adaptation will be provided in the field of population genomics and substantial resources will be provided to inform genomic approaches to crop improvement for highland environments.

% This is why your project is interesting and will help further
% knowledge in the field of mathematics. 

%How important is the proposed activity to advancing
%knowledge and understanding within its own field or across different fields?
%How well qualified is the proposer (individual or team) to conduct the project?
%(If appropriate, the reviewer will comment on the quality of prior work.)
%To what extent does the proposed activity suggest and explore creative, original,
%or potentially transformative concepts? How well conceived and organized is the
%proposed activity? Is there sufficient access to resources?  \\

\required{Broader Impacts}

As the availability of large datasets becomes increasingly common and important to society -- whether large phenotyping data from industry field trials of a crop or genome-wide-association and ancestry analysis of human genomic data -- the ability to analyze and interpret such data becomes ever more valuable. The PIs propose three important steps toward this goal.  First, they offer a public workshop aimed at enabling researchers to record and track data from large scale field trials.  Second, they will continue to develop, release, and incorporate software into their teaching with the goal of providing students  the tools to extract information from genomic data.  Third, the PIs will capitalize on their previous experience running an international exchange program to organize a summer student exchange among members of the team, giving US students experience with large field experiments and Mexican students experience analyzing genomic data. 

In addition to these goals, the PIs will continue to disseminate knowledge and resources generated in this project via open-source software, open-source publications (and preprints), deposition of useful germplasm in public repositories, and discussion of results via national and international conferences, as well as informal outreach via blogs and twitter.

% There are 4 kinds of broader impacts.
% 1. advance discovery and understanding while promoting teaching,
% training and learning
% 2. broaden the participation of underrepresented groups
% 3. disseminated broadly to enhance scientific and technological
% understanding
% 4. benefits of the proposed activity to society

