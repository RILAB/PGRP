The Information on Human Resources Developmentsection of our ABR Project Description details our 
past success in preparing graduate students for research careers and in involving undergraduates in our 
research. This will continue to be a main focus moving forward.
Undergraduate Students
Across the project, labs at four institutions have requested support for undergraduates. Undergraduate 
appointments will be of two types, full-time summer and part-time academic year. While in the past, 
summer undergraduates have mainly focused on fieldwork and genotyping, an important shift is occurring 
in our training. We will de-emphasize the mechanics of genotyping, but increase training in field biology, 
experimental design, population and quantitative genetic analysis, and bioinformatics. We will also 
attempt to increase our recruiting from engineering and computer science. During the summer, 
undergrads will learn the mechanics of maize field biology, including planting, sample collection, 
barcoding, semi-automated phenotypic data collection, pollination, and harvest. For this purpose, they are 
generally under the direct supervision of the field manager. We will also partner each undergrad with 
either a postdoc or graduate student to develop a mini-project (e.g., mapping genomic regions affecting 
traits) that may include phenotypic data collection of a new trait and bioinformatics analysis. Beginning 
with this renewal, we will have our senior bioinformatics personnel give introductions to tools via a
weekly Summer Informatics web conference (e.g., using TASSEL, R, working with large genotypic data 
sets, basic association mapping, etc.), which will be a separate meeting from the general Project Meeting 
and the Journal Club. Throughout each summer, each undergraduate will be exposed to a wide range of 
aspects of field and whole plant biology, statistical analyses, and informatics. In addition, they will 
experience first-hand the importance of team work in biology.
Part-time academic year undergraduates will continue to gain experience in both lab work and 
bioinformatics. Some will formulate and carry out small scale research projects for credit. Such 
undergrads will be mentored by a postdoc or graduate student along with the group leader, and will be 
encouraged to present their results at one of our weekly Panzea meetings and via a poster at the Maize 
meeting. We will also encourage students to continue these projects as senior honors theses. The group 
leaders will provide extensive guidance on graduate and professional opportunities.
Graduate Students
The current proposal requests funding for graduate students at three institutions. Through institutional 
support, fellowships, and training grants, far more students are likely to be participating in the project, and 
from all institutions. All institutions have longstanding traditions of preparing graduate students for 
careers in research. The requirements of their curricula ensure that students obtain necessary training in 
presentation skills, first authorship of papers, research ethics, etc. This project will enhance this
professional development training in the following ways. Presentation skills will be honed through 
presentations at internal lab meetings at least once per month and at our Panzea project web conferences 
once per semester. Furthermore, we will encourage our grad students to attend and present at a minimum 
of one scientific conference per year, and to deliver an oral presentation at least once, preferably in their 
final year. Development of informatics and statistical genetics skills will be key considerations regarding 
their course work. Hands-on experience with our large data sets and participation in our Summer 
Informatics talks will further enhance these skills. The interdisciplinary nature of our project will provide 
our grad students with broad exposure to fieldwork, lab work, bioinformatics, and population and 
quantitative genetic analysis, and to scientists from a broad variety of backgrounds.
The professional development programs listed in our Postdoctoral Mentoring Plan are generally 
available to grad students as well. We will highly encourage them to attend these. The weekly Panzea 
Journal Club, in particular, is designed for both grad students and postdocs. As described in the 
Postdoctoral Mentoring Plan, this Journal Club will have a special session on research ethics once per 
semester.