\renewcommand{\thepage}{Management Plan - Page \arabic{page} of 3}
\required{Supplementary Documentation}
\required{Management Plan}

%(A-2) Management Plan (maximum 5 pages): Projects involving multiple investigators and multiple institutions, or which include a community service component, must provide a description of the management plan for coordinating the activities of the group or management of the service aspect.
%
%This description should include plans for internal means of communication, coordination of data and information management, evaluation and assessment of progress, allocation of funds and personnel, interaction with the customers in a service project, and other specific issues relevant to the proposed activities.
%For multi-investigator proposals, a table summarizing the role of each investigator is required. The exact time commitment of each key project member should be indicated in the management plan, regardless of any request for his/her salary from NSF. For community resource projects, a timetable with yearly goals should be provided that includes benchmarks for the major anticipated outcomes and expected dates for their release.
%If the proposal includes a service component such as a multi-user facility or production and distribution of community research resources, a description of how activities within the facility will be managed, how quality will be controlled, how community input will be solicited, what methods will be used to make the community aware of the service to be rendered, and how the community will access resources to be produced, should be provided. The plan should also document institutional commitment to the facility, user fees if anticipated, and plans for long-term support after the end of the project. For a complex project, appointment of a project manager and/or administrator is strongly encouraged.
%The NSF encourages appointment of an outreach/education coordinator where appropriate. A postdoctoral fellow or a senior graduate student interested in education and outreach activities may be appointed to this role.
%

\subsection*{Communication} 

All team members will communicate on a monthly basis via a scheduled conference call. UC Davis has a ReadyTalk license allowing inexpensive web-conference hosting; other institutions will call in and can share slides, video, their desktop, and audio.  During these calls we will discuss progress, problems and solutions, as well as ways to more efficiently collaborate and coordinate among laboratories.  One member from each of two labs will present an update of their work.  Postdocs and all students will be expected to participate.

Team members will hold an annual meeting in person each year as a satellite meeting to a conference (either Plant and Animal Genome or the annual Maize Genetics Conference).  PIs not able to make the meeting will join via teleconference.  Annual meetings will consist of PIs reporting progress during the past year and goals for the upcoming year. 

\subsection*{Outreach} 

The exchange program will be coordinated between team members.  Management of visa and travel costs will be done through UC Davis, as Dr. Ross-Ibarra's program has experience with international exchange with Mexico.  

Dr. Flint-Garcia will coordinate the annual phenotyping workshop, held each year in Columbia.  The workshop will be timed to coincide with data collection at the end of the field season each year. The workshop will be advertised broadly (evoldir list-serv, maizegdb, etc.).  Attendees will be expected to pay their own travel and purchase a handheld device

\subsection*{Research}

Total research commitment to this grant for each PI will be:
\begin{itemize}
\item Angelica Cibrian Jaramillo: 5\%
\item Graham Coop: 5\%
\item Sherry Flint-Garcia: 5\%
\item Matthew Hufford: 10\%
\item Jeffrey Ross-Ibarra: 10\%
\item Ruairidh Sawers: 15\%
\end{itemize} 
Below are details of the responsibilities of each team member during each year of the grant, with initials as shown in Table \ref{tab:timeline}. Although one group will take the lead for writing publications, it is anticipated that several team members and members of their groups will be coauthors on many of these publications.
\clearpage

\begin{table}[t!]
\rowcolors{2}{gray!25}{white}
\label{tab:timeline}
\begin{center}
\caption{Summary of proposed timeline of activities showing which team members will be responsible for each objective. Details in text. Team member names are abbreviated: MBH, Matthew Hufford; JRI, Jeffrey Ross-Ibarra; SFG, Sherry Flint-Garcia; GC, Graham Coop; RS, Ruairidh Sawers; ACJ, Angelica Cibrian Jaramillo}\label{tab:timeline}
\begin{tabular}{p{3.5cm}p{2cm}p{2cm}p{2cm}p{2cm}p{2cm}}\\\toprule  
{\bf Year} & {\bf 1} & {\bf 2} & {\bf 3} & {\bf 4} & {\bf 5} \\\midrule
\ref{subsec:qtlmap} \hspace{3cm} QTL mapping 			& SFG, RS, JRI 	& SFG, MBH, RS, JRI 	& SFG, MBH, RS, JRI 	& SFG, JRI			& SFG \\\midrule
\ref{subsec:admixmap} \hspace{3cm} Admix mapping		& MBH, GC 				& MBH, GC, RS 		& MBH, GC 			& MBH, GC			& -- \\\midrule
\ref{sec:popgen} \hspace{3cm} Population genetics		& JRI, AJC, MBH, GC 		& JRI, MBH, GC  	& JRI, MBH, GC		& JRI, MBH, GC 	& JRI, MBH, GC \\\midrule
\ref{subsec:nils} \hspace{2cm} Functional \mbox{analyses} 		& RS 				& RS		& RS, SFG, MBH 			& RS, SFG 	& RS, SFG  \\\midrule
\ref{subsec:rnaseq} \hspace{2cm} RNA-seq 				& --  					& ACJ 				& JRI, ACJ, RS  			& JRI, ACJ	& JRI, ACJ \\\bottomrule
\end{tabular}
\end{center}
\end{table} 

\subsubsection*{Year 1} \hrule \vspace{0.1cm}

\paragraph{  \bf \ref{subsec:qtlmap}} SFG will generate seed of F2:3 for Mexico and F2 for S. American cross.  JRI will sequence parents of both crosses. RS will choose highland site.
\paragraph{  \bf  \ref{subsec:admixmap}} MBH will collect seed from Ahuacatitlan. GC will focus on developing methods for admix mapping.
\paragraph{  \bf  \ref{sec:popgen}} MBH will collect seed from additional admixed populations. JRI will genotype samples from highland Mexico maize. AJC and MBH will put together dataset of global highland maize. GC will work on methods for selection in admix populations.
\paragraph{ \bf \ref{subsec:nils}} RS will screen HIFs and advance the population. Initial crosses for allelic series will be performed.

\subsubsection*{Year 2} \hrule \vspace{0.1cm}

\paragraph{  \bf \ref{subsec:qtlmap}} SFG and RS will grow the mapping populations at each of 3 locales.  SFG, RS, and MBH will phenotype populations in field. SFG will genotype F2 plants. JRI will phenotype root chilling. 
\paragraph{  \bf  \ref{subsec:admixmap}} MBH will genotype samples.  RS and MBH will grow samples at two locations.  GC will begin data analysis.
\paragraph{  \bf  \ref{sec:popgen}} JRI will genotype seed from additional admix populations. MBH will genotype global highland maize collection. JRI and GC will begin data analysis of introgressed highland maize.
\paragraph{ \bf \ref{subsec:nils}} RS will advance the HIF and NIL populations.  
\paragraph{ \bf   \ref{subsec:rnaseq}} ACJ will choose NILs for RNAseq analysis

\subsubsection*{Year 3} \hrule \vspace{0.1cm}

\paragraph{  \bf \ref{subsec:qtlmap}} SFG and RS will grow a second replicate of the mapping populations at each of 3 locales.  SFG, RS, and MBH will phenotype populations in field. SFG and JRI will build map and begin QTL analysis.
\paragraph{  \bf  \ref{subsec:admixmap}} GC and MBH will complete data analysis and begin writing.
\paragraph{  \bf  \ref{sec:popgen}} JRI and GC will work on data analysis of admixed teosinte and highland Mexico maize. MBH will begin data analysis of global highland maize.
\paragraph{ \bf \ref{subsec:nils}} RS, SFG, and MBH will grow and phenotype HIFs and NILs. RS will advance both populations.
\paragraph{ \bf   \ref{subsec:rnaseq}} ACJ and RS will grow NILs and donors in highland and lowland environment.  ACJ will extract RNA. JRI will perform RNAseq library prep and sequencing.

\subsubsection*{Year 4} \hrule \vspace{0.1cm}

\paragraph{  \bf \ref{subsec:qtlmap}} SFG and JRI will perform QTL analysis
\paragraph{  \bf  \ref{subsec:admixmap}} GC and MBH will write paper. 
\paragraph{  \bf  \ref{sec:popgen}} JRI and GC will finish data analysis and begin papers for admixed teosinte and highland Mexico maize. MBH will finish analysis of global highland maize. 
\paragraph{ \bf \ref{subsec:nils}} RS will advance both populations. SFG and RS will analyze data.
\paragraph{ \bf   \ref{subsec:rnaseq}} JRI and ACJ will analyze RNAseq data.

\subsubsection*{Year 5} \hrule \vspace{0.1cm}

\paragraph{  \bf \ref{subsec:qtlmap}} SFG will write paper.
\paragraph{  \bf  \ref{sec:popgen}} MBH, JRI, and GC will write papers.
\paragraph{ \bf \ref{subsec:nils}} RS will advance both populations. SFG and RS will write paper.
\paragraph{ \bf   \ref{subsec:rnaseq}} JRI and ACJ will write paper.

