\renewcommand{\thepage}{Management Plan - Page \arabic{page} of 2}
\required{Supplementary Documentation}
\required{Management Plan}

%(A-2) Management Plan (maximum 5 pages): Projects involving multiple investigators and multiple institutions, or which include a community service component, must provide a description of the management plan for coordinating the activities of the group or management of the service aspect.
%
%This description should include plans for internal means of communication, coordination of data and information management, evaluation and assessment of progress, allocation of funds and personnel, interaction with the customers in a service project, and other specific issues relevant to the proposed activities.
%For multi-investigator proposals, a table summarizing the role of each investigator is required. The exact time commitment of each key project member should be indicated in the management plan, regardless of any request for his/her salary from NSF. For community resource projects, a timetable with yearly goals should be provided that includes benchmarks for the major anticipated outcomes and expected dates for their release.
%If the proposal includes a service component such as a multi-user facility or production and distribution of community research resources, a description of how activities within the facility will be managed, how quality will be controlled, how community input will be solicited, what methods will be used to make the community aware of the service to be rendered, and how the community will access resources to be produced, should be provided. The plan should also document institutional commitment to the facility, user fees if anticipated, and plans for long-term support after the end of the project. For a complex project, appointment of a project manager and/or administrator is strongly encouraged.
%The NSF encourages appointment of an outreach/education coordinator where appropriate. A postdoctoral fellow or a senior graduate student interested in education and outreach activities may be appointed to this role.
%

\begin{table}[]
\rowcolors{2}{gray!25}{white}
\label{tab:timeline}
\begin{center}
\begin{tabular}{p{3.5cm}p{2cm}p{2cm}p{2cm}p{2cm}p{2cm}}\\\toprule  
{\bf Year} & {\bf 2015} & {\bf 2016} & {\bf 2017} & {\bf 2018} & {\bf 2019} \\\midrule
\ref{subsec:qtlmap} \hspace{3cm} QTL mapping 			& SFG, MBH, RS, JRI 	& SFG, MBH, RS 	& SFG, MBH, RS, JRI 	& -- 			& -- \\\midrule
\ref{subsec:admixmap} \hspace{3cm} Admix mapping		& MBH 				& MBH, GC 		& MBH, GC 			& -- 			& -- \\\midrule
\ref{sec:popgen} \hspace{3cm} Population genetics		& JRI, MBH, GC 		& JRI, MBH, GC  	& JRI, MBH, GC		& JRI, GC 	& -- \\\midrule
\ref{subsec:nils} \hspace{2cm} Functional \mbox{analyses} 		& RS 				& RS, SFG		& RS, SFG 			& RS, SFG 	& RS, SFG  \\\midrule
\ref{subsec:rnaseq} \hspace{2cm} RNA-seq 				& --  					& --  				& JRI, ACJ  			& JRI, ACJ	& JRI, ACJ \\\bottomrule
\end{tabular}
\caption{Proposed timeline of activities showing which team members will be responsible for each objective.}\label{tab:timeline}

\end{center}
\end{table} 


%Yearly research goals
%Year 1:
%Aim1: Presting: Annotate and release the sequence of centromeres 2 and 5. Anchor other centromere
%contigs. Jiang: Carry out large scale CENH3 ChIP.
%Aim2a: Birchler: Stock bulk, kinetochore staining to confirm genetic interpretations. Dawe: identify B
%centromere markers for sequencing. Jiang: DNA methylation on super-stretched pachytene
%chromosomes.
%Aim2b: Dawe: Begin genetic and cytological analysis of tethered CENH3 and CENPC, build constructs
%for tethering MIS12.
%Aim3: Ross-Ibarra: collect Zea luxurians, score cSTS. Dawe: New centromere marker development
%from Z. luxurians.
%Year 2:
%Aim1: Presting: Validate and assemble FPC contigs, interpret CENH3 ChIP data, begin sequence finishing. Dawe: generate new TD markers for the centromeres targeted for completion.
%Aim2a: Birchler: FISH/immuno-analysis of progenitor, inactive and reactivated lines. Jiang: fiber- FISH/DNA methylation mapping, confirm location of B centromere TD markers using B centromere deletion derivatives.
%Aim2b: Dawe: Combination FISH / immunological assays on tethered CENH3 and CENPC lines. Determined if CENH3 recruits CENPC and vice versa, score centromere activity by cytological assays, ChiP assays on active lines.
%Aim3: Ross-Ibarra: SNP genotyping, cSTS scoring, TD analysis of Heartbreaker elements. Dawe: New marker generation from NAM lines, CRM and Cinful TD analysis of all lines.
%Year 3:
%Aim1: Presting: Validate FPC contigs, complete maps of three chosen centromeres, continue ongoing sequence finishing effort, begin sequence comparisons from other lines. Dawe: Generate TD markers as needed.
%Aim2a: Birchler: continued cytological studies, initiate RNA analysis of progenitor, inactive, and reactivated lines. Jiang: Finish fiber-FISH/DNA methylation mapping, initiate full histone modification ChIP profiles of lines.
%Aim2b: Dawe: Analyze results of combining CENH3 and CENPC together on Tether sites, full ChIP assays, determine if combined proteins recruit upper complex proteins (e.g. NDC80, MIS12, MAD2). Jiang: Fiber-DNA methylation assays on active Tether lines.
%Aim3: Ross-Ibarra: cSTS scoring and TD analysis, sequence kinetochore proteins genes. Dawe: Continue new marker development from NAM lines, CRM and Cinful assays of all lines.
%Year 4:
%Aim1: Presting: Complete BAC contigs, continue sequence finishing efforts and sequence comparisons. Jiang: Confirm contigs by fiber-FISH in oat-maize lines.
%Aim2a: Birchler: RNA analysis of progenitor, inactive, and reactivated lines. Jiang: Histone modification analysis, initiate bisulphite sequencing.
%Aim2b: Dawe: Analyze results of combining CENH3, CENPC, and MIS12 together on Tether sites, comprehensive cytological assays of mitosis and meiosis in active lines, heritability studies, full ChIP profile of lines, assay for production of centromeric Tether RNA from active lines. Jiang: DNA methylation assays on active lines as necessary.
%Aim3: Ross-Ibarra: Statistical and computational analysis. Dawe: Confirm by ChIP-TD as needed.
%Year 5:
%Aim1: Presting: Finish five centromeres and complete annotation, sequence comparisons.
%Aim2a: Birchler: Finish RNA analysis. Jiang: Finish histone modification analysis and bisulphite
%sequencing.
%Aim2b: Dawe: Finish cytological assays and heritability studies.
%Aim3: Ross-Ibarra: Finish statistical and computational analysis.
