
\renewcommand{\thepage}{f. Budget Justification - Page \arabic{page} of 3}

\required{Budget Justification}
%\begin{center}
%\emph{Maximum of 3 pages}
%\end{center}

\subsection*{Personnel}

The PI will dedicate 10\% of his time to the grant, no funding is requested for the PI, Co-PIs, or any Senior Personnel.

\subsection*{Other Personnel}
\paragraph{Graduate students} 
Funds are requested to support two graduate students each for 6 months during the academic year for each year of the project. At UC Davis, the current pay rate for doctoral students at 50\% FTE is \$27,319 during the academic year. Included is the estimated annual salary increase of 3\%.  The two students will be working on analysis of GBS data in the introgression and admix population genetic sections of AIM2, and will likely help with QTL analysis and sequencing in Aim1, and potentially RNA-seq analysis in Aim 3.

\paragraph{Technician}
Funds are requested for the first three years of the grant for a 50\% time technician (Laboratory Assistant III) to extract DNA and RNA, prepare genomic and transcriptomic sequencing libraries, and perform root chilling experiments.  The salary for this positions is set at \$36,000 (\$18,000 for 50\% time), with an annual increase of 5\%.

\subsection*{Fringe Benefits}
Fringe benefits are applied to personnel salaries using the university approved rates:
\begin{itemize}
%\item Faculty - \% in FYs 2012, 2013, and 2014
%\item Postdocs - \% in FYs 2012, 2013, and 2014
\item Graduate students - 1.3\% for all years.
%\item Undergraduate students - \% in FYs 2012, 2013, and 2014
\item Technician - 51.9\% in FYs 2015, 54.6\% in 2017, and 56.5\% in 2017
%\item Part time staff - \% in FYs 2012, 2013, and 2014
\end{itemize}

\subsection*{Equipment}

No equipment funds are requested.

\subsection*{Travel}

Travel for the PI and Co-PI Coop and one student the postdoc to 1 domestic conference each year is budgeted at \$3,000.  Travel for one of the Senior Personnel or CoPIs to participate in the field workshop is budgeted at \$1,000 each year.

Travel for Senior Personnel and members of their group to manage field experiments and phenotype is budgeted at \$12,000 each of the first 3 years. Travel for both Senior Personnel to 1 international conference each year is budgeted at \$3,000 per year.

\subsection*{Participant Support}
Our exchange program proposes to exchange two students per year between the US and Mexico.  We are requesting funds to pay for 2 exchange students per year of the grant. These funds will cover student subsistence (\$1,800 a month to include housing and subsistence) for 3 months, visa costs (\$500), and round-trip international travel (\$1,000).

\subsection*{Other Direct Costs}

 \paragraph{Materials and Supplies:}
In each of the first three years of the grant, \$15,000 is requested in materials and supplies.  \$10,000 of this is for laboratory supplies for PI Ross-Ibarra for library prep for whole genome sequencing, RNA sequencing, and DNA extraction and preparation for GBS.  This also includes funds for supplies for root chilling experiments to be done at UC Davis.  In each of the five years, \$2,500 is budgeted for standard office supplies, computer supplies (extra storage for our cluster, backup drives for lab members), and other miscellaneous expenses for Co-PI Coop and PI Ross-Ibarra. 

\paragraph{Whole genome sequencing}:
The genomes of each of the four parental lines of our QTL mapping populations will be resequenced to a depth of 20-30X using 2 lanes of paired end 150bp reads on an Illumina HiSeq 2500. Current lane costs are approximately \$2,200 per lane, and library preparations costs are approximately \$100, for a total cost of \$18,000.

\paragraph{GBS}:
Genotyping-by-sequencing will be performed for our introgression and admixture population genetic analyses. GBS will be performed at the Institute for Genomic Diversity at Cornell.  Current prices are \$60 per sample to run samples at 48-plex.  We will genotype 360 individuals for our introgression analysis in year 1 for a cost of \$21,600, and 144 individuals in year 2 for a cost of \$8,640. 

\paragraph{RNA sequencing}:
In total, RNA sequencing will be performed on 192 individuals (8 inbreds x 2 stages x 2 tissues x 2 environments x 3 replicates + 8 NILs x 2 genotypes x 2 stages x 2 environs x 3 replicates).  Cost to prepare RNA libraries in our lab are approximately \$100 per library, and sequencing costs for single-end 50bp reads at the UCD Genome Center are approximately \$1,000 per lane.  Multiplexing 12 barcodes per lane, this comes out to 32 lanes of sequence and a total cost of \$70,400.

\paragraph{Field fees:}
Fees for the field experiments in our highland and lowland field sites \ref{tab:locales} are approximately \$60,000 the first three years of the experiment to allow development of the mapping populations and two replicates of the phenotyping.  These fees include land rental and basic management (planting, watering, weeding, fertilizing), as well as station fees to hire manual labor for phenotyping.  These fees decrease to \$10,000 in the last two years of the proposal as subsequent field experiments including evaluation of NILs and RNA-seq lines, will be considerably smaller. Field fees total \$200,000 across the five years of the grant.

\paragraph{Graduate Student Tuition:}
Tuition for graduate students is charged to the project in proportion to the amount of effort the graduate student will work on the project. For a graduate student employed on the project for 9 academic months at 50\% FTE, the tuition charge is \$31,546 in FY 2015 to account for out-of-state tuition, \$17,266 in FY 2016 and increasing 5\% each subsequent year.

\paragraph{Publication Costs:}
In year two \$1,500 is requested for publication fees to an open access journal.  In subsequent years \$3,000 is requested annually.

\subsection*{Total Direct Costs}

Total direct costs for UCD come to \$874,643.  Subawards to USDA-ARS and Iowa State su to \$1,218,560.

\subsection*{Indirect Costs}
Indirect costs are calculated on Modified Total Direct Costs (Total Direct costs less graduate student fees and participant support and subaward funding beyond the first \$25,000) using F\&A rates approved by US Department of Health and Human Services. For this project, F\&A rates of 55.5\% were used from Jan. 1, 2105 through June 30, 2015, 56.5\% from July 1, 2015 through June 30, 2016, and 57\% from July 1, 2016 until the end of the project.



