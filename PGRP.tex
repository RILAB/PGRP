\documentclass[]{article}

\begin{document}

\title{Genomic Architecture of Highland Adaptation in Maize}

%Projects to be budgeted and written up:
%1) QTL mapping in high x low populations
%	SFG, RS, JRI, MBH
%2) Admixture mapping
%MBH, GC, RS
%3) Introgression: mexicana and maize
%	JRI, GC
%4) Global highland haplotypes
%	MBH, ACJ
%5) Functional Characterization: Fine\item mapping QTL & Allelic Series
%	RS, MBH, SFG
%6) Functional Characterization: RNASeq
%	ACJ
%7) Broader Impacts: Student Exchange
%	JRI
%8) Broader Impacts: Phenotyping Workshop
%	SFG

\section{Genetic architecture of highland adaptation in Zea mays}

\subsection{Questions}
\begin{itemize}
\item What is the genetic architecture of highland adaptation?
\item How much of the genetic architecture is shared between Mexico and South America?
\item  How much of the genetic architecture is shared between maize and teosinte?
\end{itemize}

\subsection{QTL mapping of highland adaptation traits in high x low mapping populations from Mexico and S. America}
\begin{itemize}
\item Phenotyping at 4 sites: 
\begin{itemize}
\item low Mexico (RS)
\item high Mexico (RS) 
\item Missouri (SFG)
\item growth chamber (MBH)
\end{itemize}
\item 960 F2:3 families; 40 checks
\item population development (RS, SFG)
\item genotyping and mapping (JRI, SFG)
\item DNA extraction, genotypes (SFG)
\item Sequence parents (JRI)
\item Mapping (SFG)
\end{itemize}

\subsection{Admixture mapping in mex/parv hybrid zone}
\begin{itemize}
\item collection trip (MBH)
\item 500 individuals from Ahuacatitlan in Mexico (MBH, RS)
\item DNA extractions, genotyping (MBH)
\item phenotyping (MBH, RS)
\item mapping (MBH/GC)
\end{itemize}

\subsection{Phenotypes}
\begin{itemize}
\item macrohairs (SFG, RS, MBH)
\item flowering time (SFG, RS, MBH)
\item tassel morphology (SFG, RS, MBH)
\item plant height every 2 weeks \& at flowering (SFG, RS, MBH)
\item biomass (SFG, RS, MBH)
\item \# ears (mz), 50k weight (mz/teo), total seed weight (mz) (SFG, RS, MBH)
\item stem/plant color (SFG, RS, MBH)
\item germination depth, temp in greenhouse (F2:3 only; MBH)
\item roots (inquire with Bloom)
\item ionomics (Ivan via letter of support)
\end{itemize}

\section{Introgression, admixture, and adaptation}
\subsection{Questions}
\begin{itemize}
\item Are introgressed loci adaptive?
\item Does evidence of introgression and natural selection correspond to QTL?
\item Are highland QTL/loci widespread in highland climes?
\end{itemize}

\subsection{Introgression and Admixture}
\begin{itemize}
\item GBS of 30 inds x 10 pops x 2 subpspecies (mex \& maize) (JRI)
\item Popgen on maize/mexicana introgression (JRI, GC)
\item GBS Additional 5 admixed parv/mex populations (50 inds. each) (JRI)
\item Introgression and adaptation in additional admix pops (JRI, GC)
\end{itemize}

\subsection{Global analysis of highland haplotypes}
\begin{itemize}
\item Occurrence of highland haplotypes/QTL/SNPs in global pops (MBH)
\item 500 worldwide accessions GBS (MBH)
\item Case study in Chihuahua (ACJ)
\end{itemize}

\section{Functional characterization of QTL}

\subsection{Questions}
\begin{itemize}
\item What do QTL/selected loci/introgressed loci do?
\end{itemize}

\subsection{Fine map pigmentation}
\begin{itemize}
\item PT x T43 NIL population (RS)
\item GBS genotyping (MBH)
\end{itemize}

\subsection{Allelic series for QTL of interest}
\begin{itemize}
\item 10 parents:
\begin{itemize}
\item 4 parents F2:3
\item mexicana TIL18
\item Palomero Toluqueño
\item 2 lowland landraces
\item 2 highland landraces
\end{itemize}
\item Cross into 3 parents for phenotyping
\begin{itemize}
\item B73 (SFG)
\item T43 (MBH)
\item CML457 (RS)
\end{itemize}
\end{itemize}

\subsection{RNAseq}
\begin{itemize}
\item Time series analysis of plants in field (ACJ):
\item 15 Lines
\begin{itemize}
\item 4 F2:3 parents
\item 1 NIL chr4,
\item 1 each mex \& parv TIL
\item B73, CML457, T43, PT
\item highland/lowland landraces used in allelic series (MBH to inbreed)
\end{itemize}
\item 12 plants per line per environment (2 pools of 6)
\item 4 stages/tissues per plant
\item 2 environments (high/low fields)
\end{itemize}

\section{Broader Impacts}

\subsection{Exchange Program}
\begin{itemize}
\item 2 students per year
\item 3-6 month stint in lab
\item Training: next-gen analysis, crossing \& phenotyping, mapping
\end{itemize}

\subsection{Phenotyping workshop}
\begin{itemize}
\item Yearly workshop at UM
\item Participants pay to purchase handheld
\item 2 day workshop provides software, training on phenotyping in maize
\end{itemize}

\subsection{Software}
\begin{itemize}
\item R code for novel analysis for admixed populations
\item Software \& pipelines on github
\end{itemize}

\subsection{Germplasm resources}
\begin{itemize}
\item NIL population \& F2:3 submitted to stock center
\end{itemize}

\end{document}