%%%%%%%%% PROJECT DESCRIPTION  -- 15 pages (including Prior NSF Support)

\required{Project Description}
\begin{center}
%\emph{Maximum of 15 pages}
\end{center}•
%The Project Description (including Results from Prior NSF Support, which is
%limited to five pages) may not exceed 15 pages. Visual materials, including charts,
%graphs, maps, photographs and other pictorial presentations are included in the
%15-page limitation. PIs be cautioned that the project description must
%be self-contained and that URLs that provide information related to the proposal
%should not be used. \\
%
%All proposals to NSF are reviewed utilizing the two merit review criteria,
%intellectual merit and broader impacts. \\
%
% The Project Description should provide a clear statement of the work 
% to be undertaken and must include: objectives for the period of the proposed 
% work and expected significance; relation to longer-term goals of the PI's 
% project; and relation to the present state of knowledge in the field, 
% to work in progress by the PI under other support and to work in progress 
% elsewhere.

\section{Rationale and Significance}

%MBH: not sure we need this paragraph. if so, we should update it.

%Maize is a natural resource of fundamental national importance, vital for food, livestock feed, and fuel production.  Maize is by far the most valuable agricultural crop in the United States: in 2010, annual production in the U.S. was worth more than $66 billion dollars (USDA 2011), or ~60\% of the value of domestic crude oil production in 2009 (EIA 2010) and nearly double the value of soybean, the second most important field crop ($39 billion, USDA 2011). With a growing human population and an increased demand for alternative fuels, maize production must keep pace. Maize yield has steadily increased over the last three decades, but continued efforts are necessary to maintain or improve this yield trajectory.  An increase in the acreage planted to maize may accommodate some of this need, but it is clear that this is not a feasible long-term solution: from 1999-2009, for example, only ~30\% of the total increase in yield can be attributed to increased acreage (USDA 2009), and every day more arable land is lost to development. Some gains in yield may of course be won by improved farming practices, but the majority of yield increase can be directly attributed to breeding efforts (Duvick 1992, 2005), and breeding must remain of central importance in order to meet increased yield demands.

%MBH: We should decide if we're going with a climate change or "adaptation" theme.  I favor the latter. our project will help understand how maize adapts to new conditions -- important for breeding lines for highland regions, dry environments, or for understanding how maize might adapt to climate change.  will also be informative about how plants in general adapt since maize is so widespread and adaptable.  what do you think about this line?

%Climate change
%Changing climatic conditions pose a serious threat to our ability to continue increasing maize yield.  Historical analyses suggests that climate change over the last 30 years has already dramatically impacted maize yields worldwide, retarding the gains from breeding and management (Lobell et al. 2011a).  While these losses have been comparatively mild in the temperate climate of the United States, the same models suggest that change of even 1ºC could negatively impact yields by as much as 5-17\% (Lobell et al. 2003, Lobell et al. 2011a). Some models even predict that U.S. maize yields could be 30-46\% below current levels by the end of the century (Schlenker and Roberts 2009).  Given predictions for long-term changes in mean temperature (Fig. 1 A-C), it is reasonable to conclude that the impacts of climate change on maize yield could translate into economic losses of billions of dollars.  Moreover, these estimates do not take into account effects of the likely yield loss in the rest of the world (Lobell et al. 2011b) on prices and supply. Substantial efforts will be needed to preserve US maize production and increase or maintain yields, and in particular breeding efforts to adapt US maize to a warming climate must be a priority (Troyer 2004).  
While future climates will undoubtedly differ in a number of important ways, we focus in this proposal on adaptation to changing temperature.  Neither of the other major changes predicted as a result of climate change — elevated CO2 and changes in rainfall — are likely to be as important for maize yields.  Though elevated CO2 may actually prove beneficial for yields of some crops, models suggest the effect on maize, a C4 plant, to be relatively minor (Lobell et al. 2011a).  Similarly, analyses of past climate change find significant negative effects of temperature on maize yield but much weaker effects of precipitation (Lobell et al. 2003, Roberts and Schlenker 2009, 2010).  Increasing temperature, on the other hand, is likely to have strong, nonlinear effects on yield (Lobell et al. 2011b, Schlenker and Roberts 2009), as temperature extremes have an inordinate impact on overall yield.  
Improved breeding methods
Novel approaches to molecular plant breeding have been proposed as a solution to changing climes (Takeda and Matsuoka 2008), and some models suggest that breeding or other technological advances could meaningfully mitigate yield loss (Li et al. 2011).  Both marker-assisted selection (MAS) and transgenic approaches have become important tools of modern plant breeding, allowing breeders to combine traits of interest without the need for additional, costly phenotyping, pedigree analysis, or (in the case of MAS) a detailed functional understanding of the molecular basis of a trait. The success of both approaches, however, depend on our ability to identify useful markers and alleles.  Genetic mapping has been effective at identifying markers for use in MAS and candidates for transgenic methods.  But traditional mapping approaches such as quantitative trait locus (QTL) and association mapping have a number of drawbacks that may limit their utility or generality (see below). 
Selection mapping can circumvent some of these problems by utilizing changes in allele frequency to identify markers that have been, or are tightly linked to, the target of historical selection. We propose to extend our previous work on selection mapping for yield to identify candidate agronomic loci (CAL) that will prove useful for molecular breeding approaches to adapt maize to changing climates. Our approach takes advantage of several millennia of adaptation of traditional maize varieties to different climate regimes, natural experiments of grand scale, using selection mapping methods to identify the targets of selection in the independent adaptation of maize to highland environments in both Mexico and South America.  Importantly, differences between highland and lowland environments mirror the predicted effects of future climate change in the US (Fig. 1 C-E).  
The resulting list of CAL, along with the bioinformatic tools we will develop, will provide new opportunities for molecular breeding, accelerating the progress of adaptation to climate change and thus ensuring continued yield increases.  Moreover, thanks to rapid advances in sequencing technology, the selection mapping approach proposed here should be easily extendible to virtually any crop species with sufficient germplasm resources; a modified method would be possible even for crops without a reference genome.  We therefore expect that our approach will prove to be an important advance in breeding methods for several crop species.

Introduction
Maize diffusion and adaptation
Despite its humble beginnings as a wild grass, maize spread rapidly across the globe to become the world’s top-producing crop.  Wild members of the genus Zea, which diverged relatively recently (Ross-Ibarra et al. 2009), are endemic to a small region stretching from northern Mexico to Central America. The direct wild ancestor of maize, Zea mays ssp. parviglumis, occurs only along the low-elevation slopes of the Sierra Madre Occidental in the southwestern corner of Mexico (Sánchez-González and Ruiz-Corral 1997).  After its domestication from ssp. parviglumis ~9,000 before present (BP; Matsuoka et al. 2002, Piperno et al. 2009), maize spread from the lowlands of Southwest Mexico (van Heerwaarden et al. 2011), rapidly diffusing across the Americas.  By ~6,000 BP, maize had adapted to the high elevations of central Mexico and spread to the lowlands of South America (Piperno 2006); by 4,000 BP maize was being grown at high altitudes in the Andes (Perry et al. 2006). After European contact with the New World, maize continued its world-wide diffusion, spreading quickly across Europe and subsequently to Asia and Africa.  Maize is now the world’s most broadly cultivated crop, currently grown on six continents in 162 countries and territories (FAOSTAT, 2009) in a distribution spanning 90° of latitude (from Chile to Canada) and more than 3000m of elevation (Tenaillon and Charcosset 2011).  During expansion to such varied regions, maize encountered — and adapted to — extremes of temperature, day length, precipitation, and soil types (Troyer 2004).  
There is clearly tremendous potential to harness the allelic variation present among maize populations for modern breeding, taking advantage of genotypes molded by selection to adapt cultivated maize to new and changing environments.  In particular, the fact that maize has independently adapted multiple times to similarly extreme environments (e.g., the highlands of the Mexican Central Plateau and the Andes) presents a unique opportunity to identify adaptive loci for use in maize breeding.  While these environments represent cooler, rather than warmer climates, identifying the loci underlying such adaptation in multiple populations should nonetheless be a powerful approach to finding CAL (see Approach below).
Genetics of adaptation
While variation in several well-known maize phenotypes (branching, glume architecture, endosperm color) is largely controlled by single loci (tb1, tga1, and y1 respectively), most complex traits have been found to be highly polygenic.  For instance, recent genome-wide association studies (GWAS) of leaf morphology, flowering time, and disease resistance have all found tens of QTL of small effect (Buckler et al. 2009, Tian et al. 2011, Kump et al. 2011).  These authors suggested that the outcrossing mating system of maize may explain the highly quantitative genetic architecture of adaptive traits in contrast to selfing species like rice and Arabidopsis where fewer loci are involved.  Adaptation to varying temperature during maize diffusion post-domestication is thus likely to have also involved a diffuse genetic architecture.  For example, tassel blasting and leaf firing, traits related to heat tolerance, have previously been shown to be polygenic in nature (Frova and Sari-Gorla 1994, Bai 2003) and temperature is known to affect many aspects of plant growth and development (e.g., seed germination, photosynthesis, respiration; Wahid et al. 2007).  

\section{Genetic architecture of highland adaptation in Zea mays}

\subsection{Questions}
\begin{itemize}
\item What is the genetic architecture of highland adaptation?
\item How much of the genetic architecture is shared between Mexico and South America?
\item How much of the genetic architecture is shared between maize and teosinte?
\end{itemize}

\subsection{QTL mapping of highland adaptation traits in high x low mapping populations from Mexico and S. America}
\begin{itemize}
\item Phenotyping at 4 sites: 
\begin{itemize}
\item low Mexico (RS)
\item high Mexico (RS) 
\item Missouri (SFG)
\item growth chamber (MBH)
\end{itemize}
\item 960 F2:3 families; 40 checks
\item population development (RS, SFG)
\item genotyping and mapping (JRI, SFG)
\item DNA extraction, genotypes (SFG)
\item Sequence parents (JRI)
\item Mapping (SFG)
\end{itemize}

\subsection{Admixture mapping in mex/parv hybrid zone}
\begin{itemize}
\item collection trip (MBH)
\item 500 individuals from Ahuacatitlan in Mexico (MBH, RS)
\item DNA extractions, genotyping (MBH)
\item phenotyping (MBH, RS)
\item mapping (MBH/GC)
\end{itemize}

\subsection{Phenotypes}
\begin{itemize}
\item macrohairs (SFG, RS, MBH)
\item flowering time (SFG, RS, MBH)
\item tassel morphology (SFG, RS, MBH)
\item plant height every 2 weeks \& at flowering (SFG, RS, MBH)
\item biomass (SFG, RS, MBH)
\item \# ears (mz), 50k weight (mz/teo), total seed weight (mz) (SFG, RS, MBH)
\item stem/plant color (SFG, RS, MBH)
\item germination depth, temp in greenhouse (F2:3 only; MBH)
\item roots (inquire with Bloom)
\item ionomics (Ivan via letter of support)
\end{itemize}

\section{Introgression, admixture, and adaptation}
\subsection{Questions}
\begin{itemize}
\item Are introgressed loci adaptive?
\item Does evidence of introgression and natural selection correspond to QTL?
\item Are highland QTL/loci widespread in highland climes?
\end{itemize}

\subsection{Introgression and Admixture}
\begin{itemize}
\item GBS of 18 inds x 10 pops x 2 subspecies (mex \& maize) (JRI)
384 at 48 plex * \$60 = \$23040
\item Popgen on maize/mexicana introgression (JRI, GC)
Identification of fine-scaled introgressed regions.
Evidence of selection against introgression (recombination)
Evidence of selection for introgressed regions (sweep signals)
Introgressed regios correlate with admix mapping signals? with QTL? Berg approach
\item GBS Additional 5 admixed parv/mex populations (50 inds. each) (JRI)
4 pops x (12 parv + 12 mex + 12 hybrids) = $8160
Ahuacatitlan and 3 more
\item Introgression and adaptation in additional admix pops (JRI, GC)
Selection for/against regions in parv/mex/admix
Parallelism across hybrid zone
Selection on quant. trait (Berg approach)
\end{itemize}

\subsection{Global analysis of highland haplotypes}
\begin{itemize}
\item Occurrence of highland haplotypes/QTL/SNPs in global pops (MBH)
\item 500 worldwide accessions GBS (MBH)
\item Case study in Chihuahua (ACJ)
\end{itemize}

\section{Functional characterization of QTL}

\subsection{Questions}
\begin{itemize}
\item What do QTL/selected loci/introgressed loci do?
\end{itemize}

\subsection{Fine map pigmentation}
\begin{itemize}
\item PT x T43 NIL population (RS)
\item GBS genotyping (MBH)
\end{itemize}

\subsection{Allelic series for QTL of interest}
\begin{itemize}
\item 10 parents:
\begin{itemize}
\item 4 parents F2:3
\item mexicana TIL18
\item Palomero Toluqueño
\item 2 lowland landraces
\item 2 highland landraces
\end{itemize}
\item Cross into 3 parents for phenotyping
\begin{itemize}
\item B73 (SFG)
\item T43 (MBH)
\item CML457 (RS)
\end{itemize}
\end{itemize}

\subsection{RNAseq}
\begin{itemize}
\item Time series analysis of plants in field (ACJ):
\item 15 Lines
\begin{itemize}
\item 4 F2:3 parents
\item 1 NIL chr4,
\item 1 each mex \& parv TIL
\item B73, CML457, T43, PT
\item highland/lowland landraces used in allelic series (MBH to inbreed)
\end{itemize}
\item 12 plants per line per environment (2 pools of 6)
\item 4 stages/tissues per plant
\item 2 environments (high/low fields)
\end{itemize}

%Projects to be budgeted and written up:
%1) QTL mapping in high x low populations
%	SFG, RS, JRI, MBH
%2) Admixture mapping
%MBH, GC, RS
%3) Introgression: mexicana and maize
%	JRI, GC
%4) Global highland haplotypes
%	MBH, ACJ
%5) Functional Characterization: Fine\item mapping QTL & Allelic Series
%	RS, MBH, SFG
%6) Functional Characterization: RNASeq
%	ACJ
%7) Broader Impacts: Student Exchange
%	JRI
%8) Broader Impacts: Phenotyping Workshop
%	SFG




\required{Broader Impacts}
\setcounter{section}{1}
% As in the project summary, broader impacts must be called out separately 
% in the project description.  You may be able to give more specific
% examples, or discuss how you've previously achieved these impacts.
% It should be similar, but not identical, to the Broader Impacts statement
% in the project summary

Also mention online presence?  twitter, online dissemination, etc.? slideshare, figshare, preprints, Haldane's sieve?

\subsection{Exchange Program} %JRI, MBH, RS
\begin{itemize}
\item 2 students per year
\item 3-6 month stint in lab
\item Training: next-gen analysis, crossing \& phenotyping, mapping
\end{itemize}

\subsection{Phenotyping workshop} %SFG
\begin{itemize}
\item Yearly workshop at UM
\item Participants pay to purchase handheld
\item 2 day workshop provides software, training on phenotyping in maize
\end{itemize}

\subsection{Software} %JRI and GC
\begin{itemize}
\item R code for novel analysis for admixed populations
\item Software \& pipelines on github
\end{itemize}

\subsection{Germplasm resources} %MBH, SFG and RS
\begin{itemize}
\item NIL population \& F2:3 submitted to stock center
\end{itemize}

\required{Results From Prior NSF Support}
% 5 pages or fewer of the 15 pages for entire description document.
% include results from NSF grants received in the past 5 years.
% If supported by more than one grant, choose the most relevant one.

% For each grant, include: 
%	(a) NSF award number, amount, dates of support 
%	(b) The title of this project
%	(c) Publications resulting from this research
%	(d) Summary of the results of the completed work
%	(e) A brief description of data samples available and other research products not described 	      elsewhere
%	(f) For renewed support, a description of the relationship between the completed and 			      proposed work

% Due to space limitations, it is often advisable to use citations rather
% than putting the titles of the publications in the body 
% of this section

%can separate pubs with semicolons if we need to to save space.

\subsubsection*{Ross-Ibarra: \#0922703: Functional Genomics of Maize Centromeres}
\emph{Intellectual merit} Centromeres are regions of the genome that organize and regulate chromosome movement, yet the biology of centromeres remains poorly understood. Co-PI Ross-Ibarra's group has focused in particular on the evolutionary genetics of centromeres. This work has demonstrated the remarkable evolutionary lability of centromere tandem repeats, but has shown that there is little evidence in maize for coevolution between centromere sequence and kinetochore proteins. Ongoing work from the Ross-Ibarra lab seeks to characterize kinetochore proteins, assess the phylogenetic evidence for longer-term coevolution, and understand patterns of centromere and genome size variation in natural populations. \\\emph{Broader impacts} Co-PI Ross-Ibarra has established and runs an international student exchange program as part of this grant. Data and result of this project have been disseminated via publications and presentations as well as deposited in the maize genetics community database \url{www.maizegdb.org}. Former trainees on the grant include Dr. Matthew Hufford (Co-PI on the current grant).
 
\begin{itemize}  \itemsep0em
\item Pyh\"aj\"arvi T, Hufford MB, Mezmouk S, Ross-Ibarra J (2013) Genome Biol \& Evol:  5:1594-1609
\item Hufford MB, Lubinsky P, Pyh\"aj\"arvi T, et al. PLoS Genetics 9(5): e1003477.
\item Melters DP, Bradnam KR, Young HA, et al. (2013) Genome Biology 14:R10
\item Kanizay LB, Pyhäjärvi T, Lowry E, et al. Heredity 110: 570-577.
\item Hufford MB, Bilinski P, Pyh\"aj\"arvi T, Ross-Ibarra J (2012) Trends in Genetics 12:606-615
\item Hufford MB, Xun X, van Heerwaarden J, et al. (2012) Nature Genetics 44:808-811
\item Chia J-M, Song C, Bradbury P, et al. (2012) Nature Genetics 44:803-807
\item Fang Z, Pyhajarvi T, Weber AL, et al. (2012) Genetics 191:883-894
\item Shi J, Wolf S, Burke J, Presting G, et al.  (2010) PLoS Biology 8: e1000327
\end{itemize}

\subsubsection*{Coop: \#1262645: Collaborative Research: ABI Innovation: Visualization And Statistics For Spatial Population Genomic Analysis. }
\$314,260, with an effective date of 05/01/13. Award Duration: 36 months.

\emph{Intellectual merit} We are developing a set of spatial statistics methods based on Gaussian random fields for the analysis of geographic population genomics data. The first method based on this approach has just been published (Bradburd et al. 2013), allowing a sound statistical framework to distinguish the effects of geographic and ecological distance on genetic isolation. \\
\emph{Broader impacts} The R package of the software has been released: \url{http://genescape.ucdavis.edu/scripts-and-code/bedassle/}, and has already been used by a many molecular ecologists. 

\begin{itemize} 
\item Bradburd, G., Ralph, P., Coop, G.  (2013) Evolution 67: 3258-3273
\end{itemize}

\subsubsection*{Flint-Garcia: \#0820619: Genetic Architecture of Maize and Teosinte}
\$ 9,823,000. 3/1/2009-2/28/2013. PI Edward Buckler, co-PIs J. Doebley, T. Fulton, S. Flint-Garcia, J. Holland, S. Kresovich, M. McMullen, Qi Sun. 

\emph{Intellectual merit} This project extends over more than a decade, and has pioneered the characterization of population genetic and evolutionary parameters of maize diversity, developed resources to connect this genetic diversity to phenotype through both association and joint linkage-association mapping, conducted fine scale analysis of domestication and agronomic QTL, and recently expanded to whole-genome analysis of diversity, evolution, and phenotype. Overall, the maize diversity project has developed a wide range of approaches and broadened understanding of the maize genome, evolution and adaptation, genetic mapping, and the agricultural improvement of maize. Over the last three years, the current iteration of the project (DBI-0820619) has successfully released and analyzed the maize Nested Association Mapping (NAM) population, collaborated on making first and second generation haplotype maps for maize, resolved domestication traits, developed a range of novel statistical approaches for association mapping, and dissected complex traits such as flowering time, kernel composition, disease resistance, height, and inflorescence and leaf morphology. Several positional cloning projects that will facilitate refinement of association mapping approaches are nearing completion.  We have also been collaborating with colleagues around the world to resequence more than 100 Zea genomes and skim sequence 20,000 maize varieties. Overall, the project has published 58 scientific papers in the last three years, including 15 in Science, Nature Genetics, and PNAS. \\
\emph{Broader impacts} The outreach program included a traveling science museum exhibit on maize diversity, evolution and genetics (seen by at least 300,000 people at five venues to date, including the famous Corn Palace in South Dakota), online Teacher Friendly Guide to the Evolution of Maize, seven Genotyping-By-Sequencing (GBS) workshops (held at primarily at Cornell but has also been held in Kenya), and training of postdocs, graduate students and undergraduates, the vast majority of which have continued in scientific careers.  Former trainees on this grant include Dr. Flint-Garcia  and Dr. Ross-Ibarra  (PIs of the current grant), only their publications are shown below.

\begin{itemize}  \itemsep0em\item Bottoms, C., Flint-Garcia, S. \& McMullen, M.D.  BMC Bioinformatics 11 Suppl 6, S28 (2010).
\item Brown, P.J. et al.  PLoS Genetics 7, e1002383 (2011).
\item Buckler, E.S. et al. Science 325, 714-8 (2009).
\item Chia, J.-M. et al. Nature Genetics 44: 803-807 (2012).
\item Cook, J.P. et al.  Plant Physiology 158, 824-834 (2012).
\item Dubois, P.G. et al. Plant Physiology 154, 173-86 (2010). 
\item Fang Z, Pyh\"aj\"arvi T, Weber AL, Dawe RK, Glaubitz JC, Sanchez Gonzalez JJ, Ross-Ibarra C, Doebley J, Morrell PL, Ross-Ibarra J. Genetics 191: 883-894. (2012)
\item Flint-Garcia, S.A., Bodnar, A.L. \& Scott, M.P.  Theoretical and Applied Genetics. 119, 1129-42 (2009).
\item Flint-Garcia, S.A., Buckler, E.S., Tiffin, P., Ersoz, E. \& Springer, N.M.  PLoS ONE 4, e7433 (2009).
\item Flint-Garcia, S.A., Dashiell, K.E., Prischmann, D.A., Bohn, M.O. \& Hibbard, B.E. Journal of Economic Entomology 102, 1317-1324 (2009).
\item Flint-Garcia, S., Guill, K. \& Sanchez-Villeda, H. Maydica 54, 375-386 (2009).
\item Hufford, M. et al. Nature Genetics 44: 808-811 (2012).
\item Hung, H.-Y. et al.  Heredity 1-10 (2011).
\item Hung H-Y, Shannon LM, Tian F, Bradbury PJ, Chen C, Flint-Garcia SA, McMullen MD, Ware D, Buckler ES, Doebley JF, \item Holland JB. PNAS 109: E1913-E1921. (2012)
\item McMullen, M.D. et al. Science 325, 737-40 (2009).
\item Morrell, P.L., Buckler, E.S. \& Ross-Ibarra, J. Nature Reviews Genetics 13, 85-96 (2011).
\item Ross-Ibarra, J., Tenaillon, M. \& Gaut, B.S. Genetics 181, 1399-413 (2009).
\item Studer, A., Zhao, Q., Ross-Ibarra, J. \& Doebley, J.  Nature Genetics 43, 1160-3 (2011).
\item Tian, F. et al.  Nature Genetics 43, 159-62 (2011).
\item van Heerwaarden, J. et al. Molecular Ecology 19, 1162-73 (2010).
\item van Heerwaarden, J., van Eeuwijk, F. a \& Ross-Ibarra, J.  Heredity 104, 28-39 (2010).
\item van Heerwaarden, J. et al. PNAS 108, 1088-1092 (2011).
\item Zhang, N, A Gur, Y Gibon, R Sulpice, S Flint-Garcia, MD McMullen, M Stitt, ES Buckler.  PLoS One 5: e9991 (2010)
\end{itemize}
