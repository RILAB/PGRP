\required{Facilities, Equipment, and Other Resources}

\setcounter{page}{1}
\renewcommand{\thepage}{Facilities, Equipment, and Other Resources - Page \arabic{page} of 1}

\begin{center}
\textbf{\large Facilities, Equipment \& Other Resources}
\end{center}

%	Identify the facilities to be used, as appropriate, indicate their capacities,
%	pertinent capabilities, relative proximity, and extent of availability to the project.
%	Use ``Other" to describe the facilities at any other performance sites listed and at
%	sites for field studies.

% \textbf{Laboratory:}
% 
% \textbf{Clinical:}
% 
% \textbf{Animal:}

\textbf{Computer:}

%	List in detail the computing capabilities of the university and the department

\textbf{Office:}

\textbf{Other:}

\textbf{MAJOR EQUIPMENT:}

%	List the most important items available for this project and, as appropriate
%	identifying the location and pertinent capabilities of each.

\textbf{OTHER RESOURCES:}

%	Provide any information describing the other resources available for the
%	project. Identify support services such as consultant, secretarial,
%	machine shop, and electronics shop, and the extent to which they will be
%	available for the project. Include an explanation of any consortium/
%	contractual arrangements with other organizations.

The PI has four standard laboratory benches as part of a shared lab space at UCD.  The shared space is the single largest lab space on campus, and provides for seamless interaction between the labs housed there.  The space currently houses three other PIs, all working on the genetics and genomics of economically important plant taxa (Dubcovsky, Neale, Dandekar). The lab is equipped with standard equipment and tools for molecular biology, including freezers and refrigeration, a shared liquid handling robot, thermal cyclers, centrifuges, gel rigs, balances, and standard molecular biology supplies.  A dedicated low-humidity refrigerator for seed storage is available through the university, and low-humidity storage cabinets for tissues and temporary seed storage are in the laboratory.

The PI occupies half of a large office suite that includes a conference room and cubicle space for 25 people.  Both macintosh and PC workstations are available for student and postdoc employees. The PI is a contributing partner in a large computer cluster, giving the lab dedicated access to 192 processors, with the opportunity for use of nearly 800 additional CPU as resources allow. Recent (2013) additions to the cluster have provided it with additional CPU as well as six new shared high-memory (512Gb RAM) nodes, one of which is dedicated to the Ross-Ibarra lab.

The PI is a faculty member of the UC Davis Genome Center, a large facility that includes bioinformatics, genotyping, metabolomics, proteomics, and expression analysis cores able to perform a variety of genomics analyses at cost for UC Davis faculty. The Genome Center also rents time on its equipment, including a bioanlyzer and library preparation robots. UC Davis has also entered into a recent partnership with BGI (the Beijing Genomics Institute) to provide additional high-throughput sequencing services via a new Sacramento-based sequencing facility.


