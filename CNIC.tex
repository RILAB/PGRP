\documentclass[]{article}
\begin{document}


\title{Genomic Architecture of Highland Adaptation in Maize}

\maketitle

\section{Genetic architecture of highland adaptation}

\subsection{Questions}
	\begin{itemize}	
		\item What is the genetic basis of highland adaptation?  
		\item How much is shared between Mexico and South America? 
		\item Between maize and teosinte? 
	\end{itemize}


\subsection{ QTL of high x low in Mexico and S. America }
		\begin{itemize}
			\item Map at 3 sites: low Mexico (RS), high Mexico (RS), Missouri (SFG)
			\item 960 F2:3 families; 40 checks
			\item population development (SFG)
			\item genotyping and mapping (JRI, SFG)
			\item DNA extraction, genotypes (SFG)
			\item Sequence parents (JRI)
			\item Mapping (SFG)
		\end{itemize}	
		
\subsection{ Admixture mapping in \emph{mex}/\emph{parv} hybrid zone  }
		\begin{itemize}
			\item 500 individuals (plant extra) from Ahuacatitlan (MBH, RS)
			\item DNA extractions (ACJ)
			\item collection trip (MBH)
			\item genotype calling (MBH, ACJ)
			\item mapping (GC)
		\end{itemize}
\subsection{ Phenotypes }
		\begin{itemize}
			\item macro hairs (SFG, RS, MBH)
			\item flowering time  (SFG, RS, MBH)
			\item tassel morphology  (SFG, RS, MBH)
			\item plant height every 2 weeks \& at flowering  (SFG, RS, MBH)
			\item biomass  (SFG, RS, MBH)
			\item \# ears (mz), 50k weight (mz/teo), total seed weight (mz)  (SFG, RS, MBH)
			\item stem/plant color  (SFG, RS, MBH)
			\item germination depth, temp in greenhouse (F2:3 only; MBH)
			\item roots (inquire with Topp, Bloom) 
	\end{itemize}

\section{Adaptation and introgression}

\subsection{Questions}
	\begin{itemize}	
		\item Are introgressed loci adaptive? 
		\item Does evidence of natural selection correspond to QTL? 
		\item Are highland QTL/loci widespread in highland climes? 
	\end{itemize}	
\subsection{ Introgression and Admixture  }
	\begin{itemize}	
		\item GBS of 30 inds x 10 pops x 2 subpspecies (mex \& maize) (JRI)
			\begin{itemize}	
				\item Popgen on maize/mexicana introgression (JRI, GC)
			\end{itemize}

		\item GBS Additional 5 admixed parv/mex populations (50 inds. each) (JRI)
			\begin{itemize}	
				\item Popgen on additional admix pops (JRI, GC)
			\end{itemize}

	\end{itemize}
\subsection{ Global analysis of highland haplotypes  }
%	\begin{itemize}	
%	\end{itemize}
	\begin{itemize}	
		\item Occurance of highland haplotypes/QTL/SNPs in global pops (MBH, ACJ)
			\begin{itemize}	
				\item 500 worldwide accessions GBS (MBH)
			\end{itemize}
		\item Case study in Chihuahua (ACJ)
		\item BAYENV2 selection on altitude in SEEDs data (RS, ACJ, GC/JRI)
	\end{itemize}

\section{ Functional characterization of QTL }
	
\subsection{ Questions } 
	\begin{itemize}
		\item What do  QTL/selected loci/introgressed loci) do?
	\end{itemize}
\subsection{ Fine map pigmentation  }
			\begin{itemize}
				\item PT x T43 NIL population (RS) 
				\item GBS genotyping (MBH)
				% T43 has Ac/Ds
			\end{itemize}
\subsection { Allelic series for QTL of interest }
	\begin{itemize}
		\item 10 parents:
			\begin{itemize}
			\item 4 parents F2:3
			\item mexicana TIL18
			\item Palomero Toluque\~no
			\item 2 lowland landraces
			\item 2 highland landraces
			\end{itemize}
		\item Cross into 3 parents for phenotyping
		\begin{itemize}
			\item B73 (SFG)
			\item T43 (MBH)
			\item CML457 (RS)
		\end{itemize}
	\end{itemize}
\subsection { RNAseq }
	\begin{itemize}
		\item Time series analysis of plants in field (ACJ)
		\item 15 Lines
			\begin{itemize}
				\item 4 F2:3 parents
				\item 1 NIL chr4, 
				\item 1 each mex \& parv TIL
				\item B73, CML457, T43, PT
				\item highland/lowland landraces used in allelic series (MBH to inbreed)
			\end{itemize}
		\item 12 plants per line per environment (2 pools of 6)
		\item 4 stages/tissues per plant
		\item 2 environments (high/low fields)
	\end{itemize}

\section { Broader Impacts }
\subsection { Exchange Program }
\begin{itemize}
\item 2 students per year
\item 3-6 month stint in other lab
\item Training: next-gen analysis, crossing \& phenotyping, mapping  
\end{itemize}
\subsection { Phenotyping workshop }
\begin{itemize}
\item Yearly workshop at UM
\item Participants pay to purchase handheld
\item 2 day workshop  provides software, training on phenotyping in maize
\end{itemize}
\subsection { Software }
\begin{itemize}
\item R code for novel analysis for admixed populations
\item Software \& pipelines on github
\end{itemize}
\subsection { Germplasm resources }
\begin{itemize}
\item NIL population \& F2:3 submitted to stock center
\end{itemize}
%\section{Catalyzing New International Collaborations (NSF)}
%Due Date:  Rolling acceptance. 

%\begin{enumerate}

%\item Need to submit request for support to NSF Plant Genome prior to submitting grant.

%\item Funds two meetings to discuss grants/preliminary data, two seasons of line development, sequencing of mapping population parents and/or GBS of populations.

%\end{enumerate}

\end{document}

